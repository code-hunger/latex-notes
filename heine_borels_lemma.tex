\documentclass{article}

\usepackage[utf8]{inputenc}
\usepackage[bulgarian]{babel}
\usepackage{amsthm}
\usepackage{amsmath}
\usepackage{amssymb}
\usepackage{enumerate}
\usepackage{enumitem}
\usepackage{titling}
\usepackage[margin=1in]{geometry}
%\usepackage[margin=10pt,a6paper]{geometry}

\newtheorem{tlemma}{Тривиална лема}
\newtheorem{lemma}[tlemma]{Лема}
\newtheorem{deff}{Дефиниция}
\newtheorem*{lemma-heine}{Лема на Хайне-Борел}
\newtheorem*{corollary}{Следствие}

\newtheoremstyle{tight}{3pt}{3pt}{}{}{\bfseries}{:}{.5em}{}

% \theoremstyle{tight}

\theoremstyle{definition}
\newtheorem{intuit}{Интуиция}

\newcommand{\powerset}[1]{\mathcal{P}(#1)}

\usepackage{multicol}

% \setlength{\droptitle}{-1.1in}

\begin{document}

\begin{lemma-heine}
    Нека в множеството  $K$ от всяка редица може да се избере сходяща подредица с граница в $K$.
    Тогава за всяко отворено покритие $\{U_\alpha\}$ същестува $\varepsilon > 0$,
    с който около всяка точка от $K$ можем да впишем кълбо с радиус $\varepsilon$ в някое $U_\alpha$.
\end{lemma-heine}

\begin{proof}
    Ще допуснем противното - тоест, че с какъвто и радиус $\varepsilon$ да опитаме,
    все ще се намери точка $x \in K$, около която да \textit{не} можем да опишем кълбо с радиус $\varepsilon$, влизащо в някое $U_\alpha$.

    Означаваме с $T(r)$ произволна точка, отговаряща на това условие за кълбета с радиус $r$.

    За да вкараме в употреба условието за сходящите подредици,
    си построяваме редица от такива точки $\{\,T(\frac{1}{m})\,\}_{m=1}^{\infty}$,
    започвайки с радиус $1$, който свиваме на всяка стъпка.

    Задачата ще ни е да достигнем противоречие, намирайки такъв неин елемент $T(r)$, около който \textit{има} кълбо с радиус $r$.

    Логическото продължение е да изберем сходяща подредица $\{\,x_t\,\}$ с граница $x_0 \in K$.
    Интересува ни някой елемент от покритието, който съдържа $x_0$.
    Да го означим с $U_\beta$.

    Отвореността на $U_\beta$ ни дава кълбо около $x_0$, което се съдържа в $U_\beta$ - да речем, $B_\delta(x_0)$.

    Щом редицата $\{ x_t \}$ \textit{клони} към $x_0$,
    значи от някъде нататък, при $t \ge t_0$ за някое $t_0$,
    \textit{всички} нейни членове влизат в това кълбо,
    че даже и в такова с половината радиус - $B_{\delta / 2}(x_0)$.

    Точката $x_{t_0}$ идва от $T(r_0)$ за някой радиус $r_0 \le \frac{\delta}{2}$. 
    Тогава е очевидно, че $B_{r_0}(x_{t_0}) \subseteq B_{\delta/2}(x_{t_0})$.

    Смело можем също да заявим, че $B_{\delta/2}(x_{t_0}) \subset B_\delta(x_0)$ (лесно се вижда). 

    Значи и $B_{r_0}(x_{t_0}) \subseteq B_\delta(x_0)$. За $B_\delta(x_0)$, обаче, казахме, че влиза в $U_\beta$, т.е. $B_{r_0}(x_{t_0}) \subset U_\beta$.

    Излезе, че около $x_{t_0}$ съществува кълбо, влизащо в някое $U_\alpha$. Самото $x_{t_0}$, обаче, е елемент на редицата $\{\,T(\frac{1}{m})\,\}_{m=1}^{\infty}$ от точки, за които не съществува такова кълбо, откъдето и идва противоречието.
\end{proof}

@TODO въпроси
защо не използваме х0, за да изкараме противоречие, вместо xt0? Какъв е радусът T(?) на x0? x0 изощо участва ли в редицата? Какво знаем за радиуса около xt0, което не знаем за този около ь0, така че да можем да изкараме противоречието?

\end{document}
