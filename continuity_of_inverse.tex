\documentclass{article}

\usepackage[utf8]{inputenc}
\usepackage[bulgarian]{babel}
\usepackage{amsthm}
\usepackage{amsmath}
\usepackage{amssymb}
\usepackage{enumerate}
\usepackage{enumitem}
\usepackage{titling}
\usepackage[margin=1in]{geometry}
%\usepackage[margin=10pt,a6paper]{geometry}

\newtheorem*{proposition}{Твърдение}
\newtheorem{deff}{Дефиниция}

\newtheoremstyle{tight}{3pt}{3pt}{}{}{\bfseries}{:}{.5em}{}

\begin{document}

\begin{proposition}
    Нека $K$ е компакт и $f: K \to R^n$ е непрекъсната инекция. 
    Тогава обратната ${f^{-1}: f(K) \to K}$ също е непрекъсната.
\end{proposition}

\begin{proof}
    Нека $y \in f(K)$ е произволно и редицата $\{y_i\} \subset f(K)$ е сходяща с граница $y$. По дефиницията на Хайне за непрекъснатост трябва да докажем, че $\{f^{-1}(y_i)\}$ също е сходяща и има граница $f^{-1}(y)$.

    Нека $x = f^{-1}(y)$ и $x_i = f^{-1}(y_i)$. 
    $K$ е компакт и $\{x_i\} \subset K$, следователно съществува сходяща подредица $\{x_{i_k}\}$ с граница някое $x' \in K$. 

    Ще проверим, че $x' = x$, независимо от подредицата, която разглеждаме. 
    Това ще означава, че $\{x_i\}$ има единствена точка на сгъстяване $x$, и тъй като е ограничена в $K$, значи е сходяща с граница $x$.

    %Непрекъснатостта на $f$ влече (Хайне), че образът на тази сходяща подредица също е сходяща редица, и то с граница $f(x')$,
    По Хайне непрекъснатостта на $f$ влече , че образът на $\{x_{i_k}\}$ също е сходяща редица, и то с граница $f(x')$,
    т.е. ${\{f(x_{i_k})\} \to f(x') \in K}$.

    $\{y_i\}$ има граница $y$, значи всяка нейна подредица, включително $\{y_{i_k}\} \equiv \{f(x_{i_k})\}_k$, има граница $y$, тоест, $f(x') = y$. 
    Но по-горе определихме $x = f^{-1}(y)$, тогава $f(x) = f(f^{-1}(y)) = y$, тоест ${f(x') = f(x)}$. 

    От инективността на $f$ имаме и $x' = x$. 
    Тогава от по-горните разсъждения вече следва, че $\{x_i\} \to x$, или ${\{f^{-1}(y_i)\} \to f^{-1}(y)}$, което и беше целта.
\end{proof}

\end{document}
