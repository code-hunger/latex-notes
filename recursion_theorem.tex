\documentclass{article}

\usepackage[utf8]{inputenc}
\usepackage[bulgarian]{babel}
\usepackage{amsthm}
\usepackage{amsmath}
\usepackage{amssymb}
\usepackage{enumerate}
\usepackage{enumitem}
\usepackage{titling}
\usepackage[margin=1in]{geometry}
% \usepackage[margin=10pt,a6paper]{geometry}

\newtheorem{tlemma}{Тривиална лема}
\newtheorem{lemma}[tlemma]{Лема}
\newtheorem{deff}{Дефиниция}
\newtheorem*{corollary}{Следствие}

\newtheoremstyle{tight}{3pt}{3pt}{}{}{\bfseries}{:}{.5em}{}

% \theoremstyle{tight}
\newtheorem{notation}{Означение}

\theoremstyle{definition}
\newtheorem{intuit}{Интуиция}

\usepackage{multicol}

\setlength{\droptitle}{-1.1in}

\title{Теорема за дефиниция по рекурсия}
\author{Александър Гудев}
\date{\vspace{-0.4in}}

\begin{document}

\maketitle

\setlength{\columnsep}{0.5cm}
\begin{multicols}{2}

\begin{deff}
    \emph{n-местна функционална операция} наричаме свойството $\Phi_n(a_1,\ldots,a_n, b)$, за което е изпълнено:
    \[
        \forall{\vec{a}}\,\forall{b_1,b_2} (\Phi_n(\vec{a},b_1) \land \Phi_n(\vec{a},b_2) \implies b_1 = b_2).
    \]
\end{deff}

\begin{notation}
    Ако $\Phi_n$ е функционална операция и е изпълнено $\Phi_n(\vec{a},b)$, пишем $F_\Phi(\vec{a})=b$.
\end{notation}

\newtheorem*{axiom}{Аксиома за замяната}
\begin{axiom}
    Ако $A_1, \ldots, A_n$ са множества и $\Phi_n $ е функционална операция, то елементите, при които $\Phi_n$ е изпълнено, също образуват множество:
    %\[\{ F_\Phi(a_1 \ldots a_n) \colon a_1 \in A_1, \ldots, a_n \in A_n \}\]
    \(\{ F_\Phi(\vec{a}) \colon \vec{a} \in A_1\!\times\!\ldots\!\times\!A_n \}\).
\end{axiom}

\begin{notation}
    Нека $(A, \le)$ е частично наредено множество и $a \in A$. Определяме:
    \begin{align*}
        A^{\le a} :=& \;\{ b \in A: b \le a \},\\
        A^{<a} :=& \; \{ b \in A: b \le a \land b \not = a \}.
    \end{align*}
\end{notation}

\begin{notation}
    Нека $f: A \rightarrow B$ е функция и $D \subseteq A$. Тогава
    \(f \upharpoonright D := \{(a,b) \in f: a \in D \}\).
\end{notation}

\begin{notation}
    Нека $A$ е множество и $f,g$ са функции с домейн $A$. Означаваме:
    \[A_{f\not=g} := \{c \in A: f(c) \not= g(c)\}.\]
\end{notation}

\begin{tlemma}
    Нека $f$ и $g$ са функции, дефинирани върху фундирано множество $A$, за които
    $A_{f\not=g}$ е непразно.

    Тогава за всеки минимален елемент $c_0$ на $A_{f\not=g}$ е изпълнено, че \(f \upharpoonright A^{<c_0} = g \upharpoonright A^{<c_0}\).
    \begin{proof}
        Нека $c \in A^{< c_0}$.
        Тъй като $c$ е по-малък от минималния елемент $c_0$ на $ D_{f \ne g} $, то $c \not \in D_{f \ne g}$, т.е. $c$ не е от елементите, в които $f$ и $g$ се различават.

        Следователно $f(c) = g(c)$, незаисимо от избора на $c$ от $A^{<c_0}$, тоест
        \({f \upharpoonright A^{<c_0} = g \upharpoonright A^{<c_0}}\).
    \end{proof}
\end{tlemma}

\begin{deff}
    Нека $\Phi_2(f,a,b)$ е функционална операция, $A$ е фундирано множество и $a \in A$.
    Казваме, че функцията \emph{$f$ е съвместима с $\Phi$ до $a$}, и пишем $\Psi_\Phi(f,a)$ при следните условия:
    \begin{enumerate}[noitemsep]
        \item $f$ е функция с $dom(f) \supseteq A^{\le{a}}$.
        \item Ограничена в домейна $A^{\le{a}}$, $f$ връща точно тази стойност, която връща и $\Phi$:
            \vspace{-6pt}
            \[{\forall a_0\!\le\!a: f(a_0) = F_\Phi( f\upharpoonright A^{< a_0},\, a_0 )}.\]
    \end{enumerate}
\end{deff}

\columnbreak

{\small
    \begin{intuit}
        Функционалната операция $\Phi$ е предикатът, който ни казва как да си построим някаква функция. Нашата цел е да докажем, че тази функция е добре дефинирана, тоест има единствена такава.

        В дефиницията за съвместимост на функция $f$ с функционална операция $\Phi$ въвеждаме за краткост на записа нов предикат $\Psi_\Phi$, чрез който проверяваме дали $f$ дава точно този резултат, който ни е даден от $\Phi$.

        От самия предикат $\Phi$ получаваме какво трябва да върне функцията $f$ за даден аргумент $a\!\in\!A$, на базата на нейната структура за всеки по-малък елемент,
        тоест използваме $\Phi$ така: ${\Phi(f \upharpoonright A^{<a},a,b)}$.
    \end{intuit}
}

\begin{lemma}
    Нека
    $A$ е фундирано множество,
    ${a,b\!\in\!A}$ са произволни,
    ${D=A^{\le a}\cap A^{\le b}}$,
    $f$ и $g$ са функции с домейн $A$,
    $\Phi_2$ e функционална операция,\linebreak
    и е изпълнено $\Psi_{\Phi}(f, a)$ и $\Psi_{\Phi}(g,b)$.

    Тогава $f \upharpoonright D = g \upharpoonright D$. % $\forall{c \in D}: f(c) = g(c)$.
\end{lemma}

\begin{proof}
    По условие имаме, че $f$ и $g$ са съвместими с $\Phi_2$ за аргументи, по-малки от съответно $a$ и $b$, и искаме да покажем, че те съвпадат в общия си домейн.

    Ще допуснем, че има елементи в $D$, за които двете функции се различават,
    и ще докажем, че множеството $D_{f \ne g}$, което ги съдържа, е празно.

    $A$ е фундирано, $A \supset D \supset D_{f \ne g}$, тогава $D_{f \ne g}$ има минимален елемент $c_0 \in D_{f \ne g} \subset D$.

    Тогава $A^{\le{c_0}} \subseteq D$, защото ${D = A^{\le a}\cap A^{\le b} = (A^{\le a})^{\le b}}$.

    По тривиалната лема следва, че ${f \upharpoonright A^{<c_0} = g \upharpoonright A^{<c_0}}$.

    По допускане имаме $\Psi_\Phi(f,a)$ и $\Psi_\Phi(g,b)$, тогава функциите са съвместими с $\Phi_2$ и в $c_0$:
    % \vspace{-5pt}
    \[ f(c_0) = F_\Phi(f\upharpoonright A^{<c_0}, c_0 ) =F_\Phi(g\upharpoonright A^{<c_0}, c_0) = g(c_0) \]

    Излиза $f(c_0) = g(c_0)$, което е противоречие с избора на $c_0 \in D_{f \ne g}$ - минмален от елементите на $D$, в които двете функции се различават.

    Следователно допускането, че $f$ и $g$ се различават в $D$, е грешно, тоест $f \upharpoonright D = g \upharpoonright D$.
\end{proof}

\begin{corollary}
    Ако $\Psi_\Phi(f,a) \land \Psi_\Phi(g,a)$, то ${f=g}$.
Следователно $\Psi_\Phi$ е функционална операция.
\end{corollary}

{\small
    \begin{intuit}
        Лема 2 ни показва, че две функции, съвместими с една и съща функционална операция до два елемента, трябва да съвпадат до по-малкия. 

        Тогава за кое да е $a\!\in\!A$ има най-много една функция, съвместима с $\Phi$ до $a$,
        тоест $\Psi_\Phi(f, a)$ е изпълнено за най-много едно $f$ по дадено $a \in A$.

        Затова и $\Psi_\Phi$ е функционална операция относно $a$ и можем да го „питаме“ коя е функцията, съвместима с $\Phi$ до $a$.
    \end{intuit}
}
\begin{lemma}
    Нека $A$ е фундирано множество, % $a\!\in\!A$,
    $\Phi$ е функционална операция,
    а $F_\Phi$ е множество от функции, съвместими с $\Phi$ за целия си домейн, всяка до някое $a \in A$.

    Тогава $\mathfrak{F} := \bigcup F_\Phi$ е функция, т.е. всички функции в $F_\Phi$ са съвместими помежду си, и $dom(\mathfrak{F}) = \bigcup \limits_{f\in F}\,dom(f)$.
\end{lemma}

\begin{proof}
    Очевидно $\mathfrak{F}$ е релация, остава да видим, че е и функция, тоест трябва да докажем:
    \[{(a,b_1) \in \mathfrak{F} \land (a,b_2) \in \mathfrak{F} \implies b_1=b_2}\]

    Нека $(a,b_1) \in \mathfrak{F}$ и $(a,b_2) \in \mathfrak{F}$.
    Тогава има функции $f_1$ и $f_2$ от $F_\Phi$, за които $f_1(a) = b_1$ и $f_2(a) = b_2$.

    По построението на $F_\Phi$ има елементи ${a_1, a_2 \ge a}$, за които $\Psi_\Phi(f_1, a_1)$ и $\Psi_\Phi(f_2, a_2)$.

    В по-частния случай е изпълнено\footnote{Тъй като е очевидно, съм си позволил да го приема наготово, но можете да го проверите от дефиницията на $\Psi_\Phi$, замествайки в нея $a_0$ с $a$-то от доказателството.} и $\Psi_\Phi(f_1, a)$ и $\Psi_\Phi(f_2, a)$,
    откъдето по следствието на лема 2\linebreak
    излиза, че $f_1 \upharpoonright A^{\le a}=f_2 \upharpoonright A^{\le a}$.

    Тогава $b_1=f_1(a)=f_2(a)=b_2$, т.е. $b_1=b_2$, следователно $\mathfrak{F}$  е функция.

    {
        \small
        А защо домейнът ѝ съвпада с обединението на домейните на изходните функции - кажи-речи, по построение\footnote{Лесно се проверява, ако вземете $(a,b)\in \mathfrak{F}$ и видите откъде е дошло; в обратната посока - взимате произволо $(a,b)$ от произволно $f\in F_\Phi$ и проверявате, че $a \in dom(\mathfrak{F})$}
        (за упражнение на читателя).
    }
\end{proof}

\begin{lemma}
    Нека $A$ е фундирано множество, $\Phi_2$ е тотална\footnote{$\Phi$ е тотална, ако винаги „връща“, т.е. когато $\forall{\vec{a}}\,\exists{b}: \Phi(\vec{a},b)$.} функционална операция.

    Тогава за всяко $a \in A$ има функция, съвместима с $\Phi_2$ до $a$: $\forall a\in A\,\exists f: \Psi(f,a)$.

\end{lemma}

\begin{proof}
    Да допуснем противното: ${\exists\,a\!\in\!A \, \nexists f: \Psi_\Phi(f,a)}$.

    Нека съвкупността от всички такива $a$-та е
    $A_{\neg{\Psi}}:=\{a\!\in\!A: \nexists f\,(\Psi_\Phi(f,a))\}$.
    По аксиомата за замяната $A_{\neg{\Psi}}$ е множество;
    искаме да видим, че то всъщност е празно.

    $A_{\neg{\Psi}} \subset A$, $A$ е фундирано, тогава $A_{\neg{\Psi}}$ има минимален елемент $c_0$.

    Целта е да покажем, че дори тогава има функция, съвместима с $\Phi_2$ до $c_0$, което ще означава, че $c_0 \notin A_{\neg{\Psi}}$, тоест $A_{\neg{\Psi}} = \emptyset$.

    Щом $c_0$ е минимален елемент в $A_{\neg{\Psi}}$,
    то за всички, по-малки от него в $A$, има такава функция.

    Нека сложим всички такива функции $f_b$ в съвкупността ${F_{<c_0} := \{f_b : \exists b<c_0 (\Psi(f_b, b)) \}}$.
    По аксиомата за замяна $F_{<c_0}$ е множество.

    \columnbreak
    Определяме \(\mathfrak{F} := \bigcup F_{<c_0}\).

    По лема 3 $\mathfrak{F}$ е функция с $dom(\mathfrak{F}) = A^{<c_0}$ и за всяко $a<c_0$ имаме функция $f \in F_{<c_0}$ с $\Psi_\Phi(f,a)$.

    Искаме да „допълним“ $\mathfrak{F}$ до $\mathfrak{F}_{c_0}$, тоест да разширим домейна до $A^{\le c_0}$, запазвайки съвместимостта с $\Phi$: $\Psi(\mathfrak{F}_{c_0}, c_0)$.

    За да запазим именно тази съвместимост, ще вземем стойността за $\mathfrak{F}_{c_0}$ в $c_0$ да бъде тази от диктуващия предикат\footnote{Такава стойност съществува, защото сме поискали $\Phi_2$ да е тотална.} $\Phi$, по който строим самата функция - тоест $F_\Phi(\mathfrak{F}, c_0)$.


    Определяме $\mathfrak{F}_{c_0} := \mathfrak{F} \cup \{(c_0, F_\Phi(\mathfrak{F}, c_0))\}$.

    От $\mathfrak{F} \subset \mathfrak{F}_{c_0}$ знаем, че $\mathfrak{F}_{c_0}(a) = F_{\Phi}(\mathfrak{F}_{c_0} \upharpoonright A^{\le a}, a)$ за всяко $a < c_0$.
    По новото построение можем да твърдим същото и за $a = c_0$.

    Следователно $\Psi_\Phi(\mathfrak{F}_{c_0}, c_0)$, т.е. намерихме функция, съвместима с $\Phi$ дори и за $c_0$, което е противоречие с допускането за $c_0 \in A_{\neg{\Psi}}$.

    Тогава $A_{\neg{\Psi}}$ е празно и ${\forall a\!\in\!A\;\exists f: \Psi_\Phi(f,a)}$.
\end{proof}

\theoremstyle{plain}
\newtheorem*{rec_theorem}{Теорема за дефиниция по рекурсия}
\begin{rec_theorem}

    Нека $(A,\le)$ е фундирано множество,
    а $\Phi_2$ е тотална функционална операция.

    Тогава съществува единствена функция $f$ с домейн $A$, съвместима с $\Phi_2$, тоест:
    \[
        \exists! f: {
            dom(f) = A \land
            \forall a\!\in\!A\,(f(a) = F_\Phi(f \upharpoonright A^{<a}, a))
        }.
    \]
\end{rec_theorem}

\begin{proof}
    Да съберем всички функции, съвместими с $\Phi_2$ до някое $a \in A$, в съвкупността ${F}_A$:
    \[ F_A := \{f_a\,|\,\exists\,a \in A(\Psi(f_a,a))\} \]

    От аксиомата за замяна знаем, че ${F}_A$ е множество.

    Нека $\mathfrak{F}_A := \bigcup F_A$. От лема 3, то е функция с $dom(\mathfrak{F}_A)=\bigcup_{f_a \in F_A} dom(f_a)$.
    Тъй като за всяко $a \in A$ имаме такова $f_a$, то $\bigcup dom(f_a) = A$, и тогава $dom(\mathfrak{F}_A) = A$.

    Нека $a\!\in\!A$ е произволно.
    По лема 4 имаме функция $f_a$, съвместима с $\Phi$ до $a$. Тогава по построението на $F_A$ следва $f_a \in F_A$, тоест $f_a \subseteq \mathfrak{F}_A$, значи ${f_a \upharpoonright A^{<a} = \mathfrak{F}_A \upharpoonright A^{<a}}$.

    От другра страна $\Psi_\Phi(f_a, a)$, значи и $\Psi_\Phi(\mathfrak{F}_A, a)$.

    От последното следва (по дефиницията на $\Psi_\Phi$), че
    $\mathfrak{F}_A(a) = F_\Phi(\mathfrak{F}_A \upharpoonright A^{<a}, a)$, независимо от избора на $a$, което и беше целта.
\end{proof}

\end{multicols}

\end{document}
